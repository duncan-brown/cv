The Laser Interferometer Gravitational Wave Observatory (LIGO) is now
operational and is presently engaged in the search for gravitational
waves from astrophysical sources~\cite{Abramovici:1992ah,Barish:1999}. The
detection of gravitational radiation is one of the most eagerly anticipated
events of twenty-first century physics, with the potential of opening a new
window on the universe and bringing unprecedented information about the
physics of strong gravitational fields~\cite{Thorne:1997ut}.

I am particularly interested in gravitational waves as an astronomical tool.
My work on the detection of inspiralling compact binaries has given me
experience with the analysis of data from gravitational wave observatories. I
plan to continue and expand this work to improve the probability of detection
and to ensure that, once a detection is made, we are in a position to extract
as much information from the waveform as possible. The interface between theory
and data analysis presents a unique opportunity to explore new physics and I
believe that this is where the future of gravitational wave astronomy lies.



\subsection{Previous Research}

Inspiralling compact binaries are the most promising sources of gravitational
radiation for the first generation of gravitational wave
interferometers~\cite{Belczynski:2002}.  On time scales of $10^7$ years, a
compact binary system loses energy by emitting gravitational waves causing its
components to spiral together. As the orbit shrinks, it circularizes and the
period decreases. With LIGO, we search for the gravitational waves that would
be emitted during the final few seconds of this inspiral. The stars orbit
hundreds of times per second at separations of tens of km before plunging
together. Binary neutron star systems could be detected with a reasonable
signal-to-noise ratio to about 20 Mpc, with an estimated rate of one every
$2.5$ years when initial LIGO reaches its design 
sensitivity~\cite{Kalogera:2000dz,Cutler:2001}, although the rate is uncertain
and could be much lower.

With the inspiral working group of the LIGO Scientific Collaboration (LSC), I
have developed a data analysis pipeline to detect the inspiral signals from
binary neutron stars (BNS). My contributions to the analysis have included the
development and implementation of code to perform the search, applying this
search code to the large data sets from the interferometers to produce
candidate events and tuning the pipeline for maximum efficiency with simulated
signals. I have also been closely involved in the efforts to analyze spurious
triggers and the statistical interpretation of results. The LSC has completed
the analysis of data from the first LIGO science run~\cite{Abbott:2003pj}.
Although no gravitational waves were found, an upper limit on the rate of BNS
coalescences has been set.

The subject of my thesis is a search for gravitational waves from binary black
hole Massive Astrophysical Compact Halo Objects (MACHOs).  Observations of
gravitational microlensing of stars in the Large Magellenic cloud suggest that
approximately 20\% of the galactic halo is composed of MACHO objects of mass
$\sim 0.5M_\odot$ and a binary system has been observed in the optical
microlensing results~\cite{Alcock:2000ph}. If MACHOs are primordial black holes
formed in the early universe then the chirp signals from their coalescence
will be detectable by LIGO~\cite{Nakamura:1997sm}. My thesis involves
searching for these signals in the data from the second and third LIGO science
runs.  In the absence of a detection, I will place a limit on the component of
the halo that may be in primordial black holes.

\subsection{Hierarchical Inspiral Searches}

As the low frequency sensitivity of the LIGO interferometers improves and the
regions of parameter space that we wish to explore grow, the computational
power required to perform a templated inspiral search increases significantly.
A search that is computationally limited may decrease the chance of making a
detection, preventing us from extracting all the astrophysical information
from that data that we may wish. For example, we may be forced to
exclude binary systems with spin.

It has been proposed that computational demands could be reduced by developing
a multi-stage or \emph{hierarchical} search
strategy~\cite{Mohanty:1996bw,Mohanty:1998eu,Tanaka:2000xy}. A template bank
that coarsely covers the parameter space of interest is generated. This is
used to trigger a search on areas of the parameter space with a refined
template bank. By using the coarse bank to guide the search to only those
areas of the parameter space where interesting signals may lie, the
processing time required to perform a deep search can be reduced. 

The inspiral code that I have developed has been designed with the goal of
hierarchical searches in mind.  Although hierarchical searches have been
studied with simulated data or small amounts of real data, performance in
large data sets has not been studied. Using the experience from flat searches
in LIGO data, I plan to study the performance of hierarchical searches in
large data sets from single and multiple interferometers.

\subsection{Extracting Information from Binary Inspiral Signals}

Once a binary inspiral has been detected, the goal is to extract as much of
the physics as possible about the event. It will be essential to assess the
statistical and systematic uncertainties of parameters extracted from the
data. The anticipated errors have been estimated in the high signal-to-noise
limit~\cite{Finn:1993xs,Chernoff:1993th,Cutler:1994}. These studies have not
used data from real interferometers, however, nor have they examined the
uncertainties obtained when digging deep in the interferometer
noise, as may be necessary with a detection in first generation
interferometers. In order to characterize the behavior of real data, large
scale Monte-Carlos will be needed. I plan to apply my experience with
Monte-Carlo simulations to the problem of parameter extraction and error
estimation.

Patrick Brady and Chuck Evans (University of North Carolina) have studied the
information that can be extracted from higher harmonics of the gravitational
wave signal from binary black hole inspiral. The quadrupolar inspiral signal
ends at a lower gravitational wave frequency than the higher harmonics of the
waveform.  The higher harmonics, therefore, have more cycles in the sensitive
band of the interferometer and so can contribute more to the signal-to-noise
ratio of the inspiral than the quadrupolar waveform.  I plan to work with
Brady and Evans to include higher
harmonics in the inspiral template families and so increase the probability of
making a detection. In the presence of a signal, we will also be able to
extract more information about the behavior of the binary system.

\subsection{Development of Search Codes for the GRID}

The purpose of GRID computing is to allow scientists to leverage large amounts
of computational power transparently and involves a significant research
component that straddles both computer science and astronomy. The inspiral
search code that I have developed is designed for the GRID environment.  My
goal is to collaborate with members of the computer science community to allow
the LSC to access as much computing power as possible for gravitational wave
astronomy.  The aim of the work is to ensure that we are not restricted in the
size of the parameter space that we can explore. My experience with tools such
as Condor and Globus are assets in this effort to leverage the power of the
GRID.

\subsection{Instrumental Origins of Inspiral Triggers}

A problem encountered when searching for inspiral signals in real
interferometer data is triggering of search algorithms by events that are of
instrumental origin, rather than from astrophysical sources. Using the tools
that I have developed I am able to quickly analyze data through the inspiral
search. I would like to work with instrumental experts to characterize the
response of search algorithms to detector artifacts. To this end, I am
enthusiastic to spend time at the observatory sites and to learn more about
the behavior and performance of gravitational wave interferometers.

