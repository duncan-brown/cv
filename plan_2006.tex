With final commissioning complete, the Laser Interferometer Gravitational-wave
Observatory (LIGO) has now reached the sensitivity goal for the first
generation of detectors. Furthermore, the commissioning of the French-Italian
VIRGO detector is progressing rapidly. We have entered an era where a
world-wide network of kilometer scale gravitational wave detectors is a
reality. A new window has been opened on the universe and the detection of
gravitational radiation is one of the most eagerly anticipated events of
twenty-first century physics.

My primary research interests are gravitational wave detection and astronomy;
the search for the first gravitational wave events from the universe and the
study of the astrophysical information that these waves contain. The promise
of gravitational wave astronomy is two-fold: Almost all of our current
understanding of the universe comes from electromagnetic observations at
various wavelengths.  Gravitational waves, being radically different in nature
from any electromagnetic observation, have the potential to bring about a
revolution in our understanding of the universe by allowing us to observe the
dynamics of the universe in a completely new way. In addition to this,
detection of gravitational waves will allow us to test General Relativity in
new arenas: the rapidly changing dynamical gravity of compact objects and the
extremely strong gravity of two black holes in collision. At present, tests of
General Relativity are confined to the weak field, slow motion realm of the
solar system; direct detection of a black hole by LIGO would be one of the
most convincing tests ever of General Relativity theory.

I am currently working on the search for gravitational waves from compact
binary coalescence, black hole quasi-normal ringdowns and the possibility of
mapping the spacetimes of extreme mass ratio inspirals to probe the strong
field limits of general relativity. Another aspect of my research is how the
recent progress in numerical relativity can be used to inform current and
future efforts in gravitational wave data analysis. By spending time at the
LIGO Observatories, I have gained a firm understanding of the relation between
analysis of the data and the detectors themselves, which complements my
background in gravitation theory. In addition to my work with ground based
gravitational wave detectors, I am collaborating with the gravitational wave
data analysis group at the NASA Jet Propulsion Laboratory to develop 
techniques for the observation of super-massive black hole inspirals in
the planned Laser Interferometer Space Antenna (LISA).

I plan to continue and expand my work towards the detection of gravitational
waves from the universe and to ensure that, once a detection is made, we
are in a position to extract as much information from those waves as possible.
The interface between relativity theory and data analysis presents a unique
opportunity to explore new physics and I believe that this is where the future
of gravitational wave astronomy lies.

\subsection{Inspiraling Compact Binaries}

Inspiraling compact binaries are one of the most promising sources of
gravitational radiation for the first generation of gravitational wave
interferometers\cite{Belczynski:2002}. On time scales of $10^7$ years, a
compact binary system loses energy by emitting gravitational waves causing its
components to spiral together. As the orbit shrinks, it circularizes and the
period decreases. With LIGO, we search for the gravitational waves that would
be emitted during the final tens of seconds of this inspiral. The stars orbit
hundreds of times per second at separations of tens of km before plunging
together. The first generation of detectors can observe binary neutron star
systems with a reasonable signal-to-noise ratio to about 20 Mpc, with an
estimated rate of one every $1.5$ years\cite{Kalogera:2003tn}, although the
rate is uncertain and could be lower. 

My previous work has focused on developing the tools needed to detect
gravitational waves from compact binary inspiral. As a member of the inspiral
working group of the LIGO Scientific Collaboration (LSC), I played a central
role in the analysis of data from the first and second science runs looking
for binary neutron star inspirals. I developed a template based matched
filtering code\cite{Allen:2005fk}, and constructed the data analysis pipelines
which analyze the detector data and post-process
results\cite{Brown:2004pv,Brown:2005zs}. While no gravitational waves were
detected in these runs, we were able place upper limits on the rate of double
neutron star coalescence better than any previous
results\cite{Abbott:2003pj,Abbott:2005pe}. In the absence of a detection, the
increased sensitivity and longer duty cycle of data presently being taken by
LIGO will allow us to place limits low enough to be comparable to those
suggested by electromagnetic observations and population synthesis models.

The coalescence of neutron star--black hole (NS-BH) binaries is believed to be
the most promising progenitor of short-hard gamma ray bursts
(GRBs)\cite{Fox:2005kv}.  The direct detection of gravitational waves
associated with a GRB would provide compelling evidence for this hypothesis,
solving the long-standing mystery of the short-hard GRB origin. The
gravitational waves from such systems are likely to be complex, however.
Coupling of the orbital angular momentum of a NS-BH binary to the spin of the
black hole causes the binary to precess. The resulting modulation of the
waveform presents significant challenges for detection, increasing the
dimension of the waveform parameter space by an order of magnitude. Building
on my experience in the development and use of data analysis pipelines, I am
currently supervising a graduate student, Diego Fazi, whose thesis project is
the implementation of the Physical Template Family for neutron star--black
hole binaries\cite{Pan:2003qt}. This template family provides waveforms that
are essentially exact for binaries of total mass $\le 15\,M_\odot$ in LIGO and
VIRGO. Using an analytic maximization of the detection statistic, the
dimensionality of the waveforms is reduced to four physical parameters: two
masses, the magnitude of the black hole spin, and the opening angle between
the spin and the orbital angular momentum. With Fazi, I plan to complete a
search for NS-BH binaries in data from the ongoing fifth LIGO science run.
This is data of unprecedented sensitivity: we will undertake both a ``blind''
search for NS-BH coalescence and a triggered search in times when GRBs have
been detected by gamma-ray detectors, such as the NASA SWIFT satellite. The use
of external triggers allows data around the trigger to be examined with
thresholds lower than the blind search, while maintaining a comparable false
alarm rate.

I am also a member of the LSC inspiral review committee, which is responsible
for the approval of LIGO results before public release. As a reviewer, I am
closely involved in all aspects of the work of the inspiral group, including
the searches for binary black holes and binary neutron stars in data from the
fifth LIGO science run.

\subsection{Probing the Spacetime Geometry of Strong Field Objects}

A major component of my current research is the study of the inspiral and
coalescence of neutron star--intermediate mass black hole binaries detectable
by the ground based network of detectors. In recent years, evidence from
ultra-luminous X-ray sources and the dynamics of globular clusters suggests
that there may exist a population of intermediate mass black holes (IMBHs) with
masses in the range $M \sim 10^2$ -- $10^4\,M_\odot$\cite{Miller:2003sc}.
The motivation to search for such systems in ground based detectors is that
the mass ratio of the binary components may be high enough that the neutron
star will act as a probe of the gravitational field of the larger black
hole.  Rate estimates of neutron star--IMBH capture in globular clusters
suggest that the coalescence rate may be as high as 10 per year for the
Advanced LIGO detectors.

With other members of the Theoretical Astrophysics and Relativity Group at
Caltech, I am studying the astrophysics that will be accessible upon
detection, and the data analysis techniques that may be employed to search for
such intermediate mass ratio inspirals. I am particularly interested in
measurement of the multipole structure of the spacetime using LIGO, and the
possibilities for determining the spacetime geometry with sufficient accuracy
to test the no-hair theorem.  Ryan has shown that, under the restriction of
nearly circular and nearly equatorial orbits, the multipolar structure of the
spacetime may be extracted from the gravitational waves emitted by the
inspiral\cite{Ryan:1995wh}.  While Ryan proved \emph{in principle} that the
multipolar structure of spacetime is encoded in the gravitational waves, he
did not provide an algorithm to extract this information \emph{in practice.} A
major direction of my future research is to explore ways in which Ryan's
methods may be used to extract the information encoded in the LIGO and LISA
data streams.

\subsection{Numerical Relativity and the Extraction of Information from Black
Hole Binaries}

The recent developments in numerical relativity are very encouraging for
gravitational wave
astrophysics\cite{Pretorius:2005gq,Campanelli:2005dd,Baker:2006yw}. Progress
in binary black hole simulations holds promise for improving gravitational
wave searches and maximizing the amount of astrophysical information that one
could extract from a detection of a binary inspiral signal. I am currently
collaborating with the Caltech/Cornell numerical relativity group to determine
how information from numerical simulations can be used to improve searches for
binary black hole coalescence in LIGO data. Our current work is focused on
improving the current generation of binary black hole searches, by using data
from numerical waveforms to test and enhance data analysis techniques. I plan
to continue this collaboration and expect it to grow as numerical relativity
informs data analysis and data analysis guides numerical relativity. I firmly
believe the interface between relativity theory and gravitational wave
observation will be an exciting place in the near future. I plan to be in
a position to make full use of my experience and knowledge to learn as much
about the universe as possible.

Once a binary inspiral has been detected, the goal is to extract as much of
the physics as possible about the event. It will be essential to assess the
statistical and systematic uncertainties of parameters extracted from the
data. The anticipated errors have been estimated in the high signal-to-noise
limit\cite{Finn:1993xs,Chernoff:1993th,Cutler:1994}. These studies have not
used data from real interferometers, however, nor have they examined the
uncertainties obtained when digging deep in the interferometer noise, as may
be necessary with a detection in first generation interferometers. In order to
characterize the behavior of real data, large scale Monte-Carlos will be
needed. I plan to apply my experience with Monte-Carlo simulations to the
problem of parameter extraction and error estimation.

\subsection{Black Hole Ringdowns}

If the final state of a binary inspiral is a black hole it is expected to be
a perturbed black hole which will radiate away the perturbations as
gravitational waves. At late times, the distorted black hole will be
sufficiently similar to a Kerr black hole that black hole perturbation theory
can be used to model the waveforms\cite{Teukolsky:1973ha}. These ringdown
waveforms are a superposition of quasi-normal modes, whose frequency and
damping time uniquely depend on the mass and spin of the black hole. This
uniqueness is directly related to the no-hair theorem and a reliable detection
of ringdown waveforms would provide a key test of general relativity in the
strong field regime\cite{Dreyer:2003bv}. I am presently co-mentoring a
graduate student, Lisa Goggin, who is leading the search for ringdown signals
from black holes in LIGO data. Although the rate of ringdowns in the universe
is unknown, they are an extremely bright source for ground based detectors.
For a $200\,M_\odot$ black hole with spin $a = 0.9$, the ringdown signal could
be observed with reasonable signal-to-noise ratios at distances up to several
hundred megaparsecs. I plan to continue to play a key role in the ringdown
search and the black hole spectroscopy that a detection would herald.

\subsection{GRID Computing and Gravitational Wave Physics}

The purpose of GRID computing is to allow scientists to leverage large amounts
of computational power transparently and involves a significant research
component that straddles both computer science and astronomy. The search codes
that I have developed are designed for the GRID environment.  My goal is to
collaborate with members of the computer science community to allow the LSC to
access as much computing power as possible for gravitational wave astronomy.
The aim of this work is to ensure that we are not restricted in the size of the
parameter space that we can explore. My experience with tools such as Condor
and Globus are assets in this effort to leverage the power of the GRID.

As a graduate student, I played a key role in the design and construction of
the Medusa Beowulf cluster at the University of Wisconsin--Milwaukee. Medusa
has served as one of the key computing centers for the LSC. The knowledge I
gained from this experience will allow me to be a leader in the design and
implementation of the next generation of computing clusters used for
gravitational wave data analysis.

\subsection{Conclusion}

Gravitational wave detection is a new and exciting field. We are now studying
data of unprecedented sensitivity which may contain the first gravitational
waves to be seen; waves which will carry unique information about the nature
of the universe, probing regions hitherto inaccessible to electromagnetic
observations. I plan to build on my past research to position myself at the
forefront of a new field of physics: gravitational wave astronomy. The
objective of my research is to use the information we will obtain from
observations of gravitational waves to learn about the generation of waves in
distant sources and the nature of gravity itself. I look forward to being a
leader in this adventure.
