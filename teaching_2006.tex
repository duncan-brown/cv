I consider working with students to be extremely important and as such I am
currently mentoring two graduate students. I have worked with graduate student
Lisa Goggin since my arrival at Caltech and encouraged her to take the lead on
the LSC search for binary black hole ringdowns. She is now a fifth year
student and is making excellent progress towards graduating in 2007. I also
mentor Diego Fazi, a second year graduate student, who is working on the
search for neutron star--black hole inspirals. He is presently learning the
techniques of theoretical and practical data analysis and is commissioning a
new type of search on LIGO data. In the summers of 2005 and 2006, I
participated in the LIGO Research Experience for Undergraduates (REU) program.
I devised a short-term research project for the students and guided them
over a ten week period, leading to a final report. Although it is
challenging to provide a project that can completed in a short period of time
by a student with little background experience, I found this to be a very
positive experience. The student I supervised in 2005 is currently applying
for graduate school to pursue a PhD in LIGO data analysis.

I find my interactions with both graduate and undergraduate students to be
stimulating and productive. When guiding students through difficulties, I
invariably learn something new about a problem. It is particularly rewarding
to see students begin to develop and pursue their own ideas and mature into
scientists ready to conduct independent research. Gravitational wave data
astrophysics is an extremely fertile field for students to perform research
and I have many ideas for projects at both the graduate and undergraduate
level. Data analysis has many open computational problems that are ideal for
undergraduate students, as they do not require a deep background in
gravitational wave physics for the student to begin making progress. As the
student progresses they can gain an understanding of the underlying physics
which will prepare them for more advanced study.  There are also many
opportunities for graduate research in gravitational wave astrophysics, from
theoretical projects to develop new detection techniques, practical analysis
of data from the detectors and computational physics, to astrophysical
interpretation of results and observations.

I look forward to continuing to work with and inspire students. In addition to
leading discussion sections for physics and engineering majors, during my time
as a graduate teaching assistant I was able to gain experience preparing
classes and lecturing to large, diverse groups of students. I consistently
gained excellent evaluations from my students. 

Finally, I believe that outreach to both students and the public is also an
essential part of my work. While at Caltech, I have accepted invitations to
speak about LIGO and gravitational wave astrophysics to undergraduate physics
majors at Whittier College and to students and the general public at Riverside
Community College.  I enjoyed these opportunities immensely and hope that I
managed to convey some of the excitement of the field to these audiences.
